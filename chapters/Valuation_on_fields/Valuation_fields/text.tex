\section{Valuation fields}

\subsection{Absolute values and completion}

    \begin{definition}\label{def:valuation_field}
        Let \(\kk\) be a field.
        An \emph{absolute value} on \(\kk\) is a function \(\|\cdot|:\kk\to\bbR_{\ge0}\) satisfying the following properties for all \(x,y\in\kk\):
        \begin{enumerate}
            \item \(\|x\|=0\) if and only if \(x=0\);
            \item \(\|xy\|=\|x\|\cdot\|y\|\);
            \item \(\|x+y\|\leq\|x\|+\|y\|\).
        \end{enumerate}
        A field \(\kk\) equipped with an absolute value \(\|\cdot\|\) is called a \emph{valuation field}.
    \end{definition}

    \begin{remark}\label{rmk:additive_and_multiplicative_valuation_on_a_field}
        Let \(\kk\) be a field.
        Recall that a (additive) valuation on \(\kk\) is a function \(v: \kk^\times \to \bbR\) such that
        \begin{itemize}
            \item \(\forall x,y \in \kk^\times, v(xy) = v(x) + v(y)\);
            \item \(\forall x,y \in \kk^\times, v(x + y) \geq \min\{v(x), v(y)\}\).
        \end{itemize}
        We can extend \(v\) to the whole field \(\kk\) by defining \(v(0) = +\infty\).
        Fix a real number \(\varepsilon \in (0,1)\).
        Then \(v\) induces an absolute value \(|\cdot|_v: \kk \to \bbR_+\) defined by \(|x|_v = \varepsilon^{v(x)}\) for each \(x \in \kk\).

        The valuation \(v\) defined above is called an \emph{additive valuation}.
        And an absolute value \(|\cdot|\) on \(\kk\) is called a \emph{multiplicative valuation}.
        In this note, the term \emph{valuation} may refer to either an additive valuation or a multiplicative valuation, depending on the context.
    \end{remark}

    \begin{example}\label{eg:trivial_absolute_value_on_any_fields}
        Let \(\kk\) be a field.
        The \emph{trivial absolute value} on \(\kk\) is defined as
        \[ \|x\| := \begin{cases}
            0, & x = 0; \\
            1, & x \neq 0.
        \end{cases} \]
    \end{example}

    \begin{definition}\label{def:valuation_group_of_a_valuation_field}
        The \emph{(multiplicative) valuation group} of a valuation field \((\kk,\|\cdot\|)\) is defined as the subgroup of \(\bbR_{>0}\) given by
        \[ |\kk^\times| := \{\|x\|\colon x \in \kk^\times\}. \]
        We use the notation \(\sqrt{|\kk^\times|}\) to denote the set \(\{\|x\|^{1/n}\colon x \in \kk^\times, n \in \bbZ_{>0}\}\).
    \end{definition}

    Note that an absolute value \(\|\cdot\|\) is non-trivial if and only if its valuation group \(|\kk^\times|\) is not equal to \(\{1\}\).

    \begin{definition}\label{def:complete_valuation_field}
        Let \((\kk,\|\cdot\|)\) be a valuation field.
        We say that \(\kk\) is \emph{complete} if the metric \(d(x,y) := \|x - y\|\) makes \(\kk\) a complete metric space.
    \end{definition}

    \begin{lemma}\label{lem:completion_of_valuation_field}
        Let \((\kk,\|\cdot\|)\) be a valuation field and \((\widehat{\kk},\|\cdot\|)\) its completion as a metric space.
        Then the operations of addition and multiplication on \(\kk\) can be extended to \(\widehat{\kk}\) uniquely, making \((\widehat{\kk},\|\cdot\|)\) a complete valuation field containing \(\kk\) as a dense subfield.
    \end{lemma}
    \begin{proof}
        Note that the operations of addition and multiplication on \(\kk\) are uniformly continuous with respect to the metric \(d(x,y) = \|x - y\|\).
        Thus they can be extended to \(\widehat{\kk}\) uniquely.
    \end{proof}

    \begin{proposition}\label{prop:non-trivial_complete_field_is_uncountable}
        Let \((\kk,\|\cdot\|)\) be a complete valuation field with non-trivial absolute value.
        Then \(\kk\) is uncountable.
    \end{proposition}
    \begin{proof}
        Since the absolute value \(\|\cdot\|\) is non-trivial, we can construct a sequence \(\{x_n\}_{n=1}^\infty \subseteq \kk\) inductively such that \(\|x_n\| < \|x_{n-1}\|/2\) for any \(n\geq 1\) and \(\|x_0\| < 1\).
        Then there is an injective map from \(\bbN^{\{0,1\}}\) to \(\kk\) defined by
        \[ (a_n)_{n=1}^\infty \mapsto \sum_{n=1}^\infty a_n x_n, \quad a_n \in \{0,1\}. \]
        Since \(\|x_n\| < 2^{-n}\), the series \(\sum_{n=1}^\infty a_n x_n\) converges in \(\kk\).
        Note \(\|x_n\| > \|\sum_{m\geq n} x_m\|\) for each \(n\), we have that the map is injective.
        Thus \(\kk\) is uncountable.
    \end{proof}

    Unlike the real number field \(\bbR\), even a valuation field is complete, we can not expect the theorem of nested intervals to hold.

    \begin{definition}\label{def:spherically_complete}
        A valuation field \((\kk,\|\cdot\|)\) is called \emph{spherically complete} if every decreasing sequence of closed balls in \(\kk\) has a non-empty intersection.
    \end{definition}

    \begin{example}\label{eg:p-adic_complex_field_is_not_spherically_complete}
        The field \(\bbC_p\) of \(p\)-adic complex numbers is not spherically complete, see \Yang{to be added.}
    \end{example}

    \begin{example}\label{eg:usual_absolute_value_on_Q}
        Let \(|\cdot|_\infty\) be the usual absolute value on the field \(\bbQ\) of rational numbers.
        Then \((\bbQ, |\cdot|_\infty)\) is a valuation field.
        Its completion is the field \(\bbR\) of real numbers equipped with the usual absolute value.
    \end{example}

    \begin{example}\label{eg:p-adic_absolute_value_on_Q}
        Let \(p\) be a prime number.
        For any non-zero rational number \(x \in \bbQ\), we can write it as \(x = p^n \frac{a}{b}\), where \(n \in \bbZ\) and \(a,b \in \bbZ\) are integers not divisible by \(p\).
        The \emph{\(p\)-adic absolute value} on \(\bbQ\) is defined as
        \[ |x|_p := \begin{cases}
            0, & x = 0; \\
            p^{-n}, & x = p^n \frac{a}{b} \text{ as above}.
        \end{cases} \]
        Then \((\bbQ, |\cdot|_p)\) is a valuation field.
        Its completion is the field 
        \[ \bbQ_p = \left\{\sum_{n = k}^{+\infty} a_n p^n \colon k \in \bbZ, a_n \in \{0,1,\ldots,p-1\}\right\} \]
        of \(p\)-adic numbers equipped with the \(p\)-adic absolute value; see \Yang{to be added.}.
    \end{example}
    
    \begin{definition}\label{def:archimedean_and_non-archimedean_valuation}
        Let \((\kk,\|\cdot\|)\) be a valuation field.
        We say that \(\kk\) is \emph{non-archimedean} if its absolute value \(\|\cdot\|\) satisfies the \emph{strong triangle inequality}:
        \[ \|x+y\|\leq\max\{\|x\|,\|y\|\},\quad\forall x,y\in\kk. \]
        Otherwise, we say that \(\kk\) is \emph{archimedean}.
    \end{definition}

    \begin{example}\label{eg:archimedean_absolute_value_and_additive_valuation}
        Let \(v\) be an additive valuation on a field \(\kk\).
        Then the induced absolute value \(|\cdot|_v\) as in \cref{rmk:additive_and_multiplicative_valuation_on_a_field} is non-archimedean.

        The converse is also true: if \((\kk, |\cdot|)\) is a non-archimedean valuation field, 
        then the function \(v: \kk^\times \to \bbR\) defined by \(v(x) = -\log |x|\) is an additive valuation on \(\kk\).
    \end{example}

    \begin{proposition}\label{prop:archimedean_iff_integer_are_unbound}
        Let \((\kk,|\cdot|)\) be a valuation field.
        Then \(\kk\) is archimedean if and only if the set \(\{|n \cdot 1|\colon n \in \bbZ\}\) is unbounded.
    \end{proposition}
    \begin{proof}
        Sufficiency is obvious.
        \Yang{To be added.}
    \end{proof}

\subsection{Places on a field}

    \begin{definition}\label{def:equivalent_of_absolute_values}
        Let \(\kk\) be a field.
        Two absolute values \(\|\cdot\|_1\) and \(\|\cdot\|_2\) on \(\kk\) are said to be \emph{equivalent} if there exists a real number \(c \in (0,\infty)\) such that
        \[ \|x\|_1 = \|x\|_2^c, \quad \forall x \in \kk. \]
    \end{definition}

    Note that equivalent absolute values induce the same topology on the field \(\kk\).
    Moreover, the following lemma shows that the converse is also true.

    \begin{lemma}\label{prop:equivalent_of_absolute_values_and_topology_and_unit_disk}
        Let \(\kk\) be a field and \(\|\cdot\|_1\), \(\|\cdot\|_2\) be two absolute values on \(\kk\).
        Then the following statements are equivalent:
        \begin{enumerate}
            \item \(\|\cdot\|_1\) and \(\|\cdot\|_2\) are equivalent;
            \item \(\|\cdot\|_1\) and \(\|\cdot\|_2\) induce the same topology on \(\kk\);
            \item The unit disks \(D_1 = \{x \in \kk\colon \|x\|_1 < 1\}\) and \(D_2 = \{x \in \kk\colon \|x\|_2 < 1\}\) are the same.
        \end{enumerate}
    \end{lemma}
    \begin{proof}
        The implications (a) \(\Rightarrow\) (b) is obvious.
        Now we prove (b) \(\Rightarrow\) (c).
        For any \(x \in D_1\), we have \(x^n \to 0\) as \(n \to \infty\) under the absolute value \(\|\cdot\|_1\) and thus under \(\|\cdot\|_2\).
        Therefore, \(\|x\|_2^n \to 0\) as \(n \to \infty\), which implies that \(\|x\|_2 < 1\), i.e., \(x \in D_2\).
        Similarly, we can prove that \(D_2 \subseteq D_1\).

        Finally, we prove (c) \(\Rightarrow\) (a).
        If \(\|\cdot\|_1\) is trivial, then \(D_1 = \{0\}\) and thus \(\|\cdot\|_2\) is also trivial.
        In this case, they are equivalent.
        Suppose that both \(\|\cdot\|_1\) and \(\|\cdot\|_2\) are non-trivial.
        Pick any \(x,y \notin D_1 = D_2\).
        Then there exist real numbers \(\alpha, \beta > 0\) such that \(\|x\|_1 = \|x\|_2^\alpha\) and \(\|y\|_1 = \|y\|_2^\beta\).
        Suppose the contrary that \(\alpha \neq \beta\).
        Consider the domain \(\Lambda \subseteq \bbZ^2\) defined by
        \[ \begin{cases}
            n \log \|x\|_2 < m \log \|y\|_2; \\
            n \alpha \log \|x\|_2 > m \beta \log \|y\|_2.
        \end{cases} \]
        Since \(\alpha \neq \beta\), the two lines defined by the equalities are not parallel.
        Thus \(\Lambda\) is non-empty.
        Pick \((n,m) \in \Lambda\) and set \(z := x^n y^{-m}\).
        Then we have \(\|z\|_2 < 1\) and \(\|z\|_1 > 1\), a contradiction.
    \end{proof}

        
    \begin{theorem}[Ostrowski]\label{thm:classification_of_archimedean_complete_fields}
        Let \((\kk,\|\cdot\|)\) be an archimedean complete valuation field.
        Then \(\kk\) is isomorphic to either the real number field \(\bbR\) or the complex number field \(\bbC\) equipped with the usual absolute value.
        % \Yang{To be revised.}
    \end{theorem}
    \begin{proof}
        \Yang{To be added.}
    \end{proof}

    \begin{definition}\label{def:place_on_a_field}
        Let \(\kk\) be a field.
        A \emph{place} on \(\kk\) is an equivalence class of non-trivial absolute values on \(\kk\).
        An archimedean (resp. non-archimedean) place is also called an \emph{finite} (resp. \emph{infinite}) place.
        We denote the set of all places (resp. finite, infinite places) on \(\kk\) by \(M_\kk\) (resp. \(M_\kk^f\), \(M_\kk^\infty\)).
        
        If \(\bfl \subset \kk\) is a subfield, we denote by \(M_{\kk/\bfl}\) (resp. \(M_{\kk/\bfl}^f\), \(M_{\kk/\bfl}^\infty\)) the set of all places (resp. finite, infinite places) on \(\kk\) which are trivial on \(\bfl\).
    \end{definition}

    \begin{example}\label{eg:relative_place_set_of_C_t}
        Let \(\kk = \bbC(t)\) and \(\bfl = \bbC\).
        Then I claim that
        \[ M_{\bbC(t)/\bbC} \cong \{\text{ prime divisors on } \bbP_\bbC^1\}. \]
        For each prime divisor \(P\) on \(\bbP_\bbC^1\), we can define an additive valuation \(\Mult_P: \bbC(t)^\times \to \bbZ\) by assigning to each non-zero rational function \(f \in \bbC(t)^\times\) its multiplicity at \(P\).
        Fix a real number \(\varepsilon \in (0,1)\).
        Then we obtain a multiplicative valuation (absolute value) \(|\cdot|_P\) on \(\bbC(t)\) as in \cref{rmk:additive_and_multiplicative_valuation_on_a_field}.
        It is easy to check that the absolute value \(|\cdot|_P\) is trivial on \(\bbC\) and that different prime divisors give rise to inequivalent absolute values.

        Conversely, given any non-trivial absolute value \(|\cdot|\) on \(\bbC(t)\) which is trivial on \(\bbC\), by \cref{prop:archimedean_iff_integer_are_unbound,eg:archimedean_absolute_value_and_additive_valuation}, 
        the absolute value \(|\cdot|\) is given by an additive valuation \(v: \bbC(t)^\times \to \bbR\).
        Let \(\calO_v\) be the valuation ring of \(v\) and \(\frakm_v\) its maximal ideal.
        Then \(t \in \calO_v\) or \(t^{-1} \in \calO_v\).
        Without loss of generality, we assume that \(t \in \calO_v\).
        Since \(|\cdot|\) is trivial on \(\bbC\), we have \(\bbC[t] \subseteq \calO_v\).
        And we have \(\frakm_v \cap \bbC[t] \neq 0\) since otherwise \(v\) is trivial on \(\bbC[t]^\times\) and thus on \(\bbC(t)^\times\).
        It follows that the image of \(\bbC[t]\) under the quotient map \(\calO_v \to \calO_v/\frakm_v\) is \(\bbC\).
        This gives a point \(P \in \bbA_\bbC^1 \subseteq \bbP_\bbC^1\).
        Then \(v\) is different from the additive valuation \(\Mult_P\) by a positive scalar multiple via looking at the values on \(\bbC[t]\).
        Thus we have established the claimed bijection.
    \end{example} 

    \begin{theorem}[Ostrowski]\label{theorem:ostrowski_theorem}
        Every nontrivial absolute value on \(\bbQ\) is equivalent to either the usual absolute value \(|\cdot|_\infty\) or a \(p\)-adic absolute value \(|\cdot|_p\) for some prime number \(p\).
        In other words, 
        \[ M_\bbQ = \{|\cdot|_\infty\} \cup \{| \cdot |_p : p \text{ is a prime number}\}. \]
    \end{theorem}
    \begin{proof}
        \Yang{To be added.}
    \end{proof}

    \begin{remark}\label{rmk:normalized_place_on_Q}
        For every non-archimedean place \(v\) on \(\bbQ\) corresponding to a prime number \(p\), 
        we choose the unique normalized absolute value \(|\cdot|_v\) in the class \(v\) such that \(|p|_v = 1/p\).
        For the archimedean place \(v\) on \(\bbQ\), we choose the usual absolute value \(|\cdot|_v = |\cdot|_\infty\).
        Unless otherwise specified, we always use the normalized absolute values on \(\bbQ\).
    \end{remark}

    \begin{remark}\label{rmk:product_formula_on_Q}
        For any non-zero rational number \(x \in \bbQ^\times\), one can easily check the \emph{product formula}:
        \[ \prod_{v \in M_\bbQ} |x|_v = 1. \]
        This can be viewed as an arithmetic analogue of the fact on \(\bbP_\bbC^1\) that 
        \[ \sum_{P \in \bbP^1(\bbC)} \Mult_P(f) = 0 \]
        for any non-zero rational function \(f \in \bbC(t)^\times\).
        Indeed, fix a real number \(\varepsilon \in (0,1)\).
        Then by \cref{eg:relative_place_set_of_C_t}, above fact can be rewritten as
        \[ \prod_{P \in \bbP_\bbC^1} |f|_P = 1. \]
    \end{remark}
    
    \begin{theorem}[Artin-Whaples approximations]\label{theorem:artin_whaples_approximations}
        Let \(\kk\) be a field and \(v_1, v_2, \ldots, v_n \in M_\kk\) be pairwise distinct places on \(\kk\).
        For each \(i \in \{1,2,\ldots,n\}\), let \(x_i \in \kk\) and \(\varepsilon_i > 0\).
        Then there exists an element \(x \in \kk\) such that
        \[ |x - x_i|_{v_i} < \varepsilon_i, \quad \forall i \in \{1,2,\ldots,n\}. \]
        In particular, the image of the diagonal embedding
        \[ \kk \to \prod_{i=1}^n \kk_{v_i} \]
        is dense, where \(\kk_{v_i}\) is the completion of \(\kk\) with respect to the place \(v_i\).
        % \Yang{To be checked.}
    \end{theorem}
    \begin{proof}
        \Yang{To be added.}
    \end{proof}



