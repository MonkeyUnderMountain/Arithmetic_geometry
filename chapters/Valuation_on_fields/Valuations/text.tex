\section{Valuations}

    Let \(\kk\) be a field.
    Usually, we consider \(\kk\) to be a number field or a function field.

\subsection{Definition}

    \begin{definition}\label{def:valuation_on_fields}
        An \emph{absolute value} or a \emph{valuation} on \(\kk\) is a function \(|\cdot| : \kk \to \bbR_{\geq 0}\) satisfying the following properties:
        \begin{itemize}
            \item \(|x| = 0\) if and only if \(x = 0\);
            \item \(|xy| = |x| \cdot |y|\) for all \(x, y \in \kk\);
            \item \(|x + y| \leq |x| + |y|\) for all \(x, y \in \kk\).
        \end{itemize}
    \end{definition}

    \begin{remark}\label{rmk:additive_and_multiplicative_valuation_on_a_field}
        % Let \(\kk\) be a field.
        Recall that a \emph{additive valuation} on \(\kk\) is a function \(v: \kk^\times \to \bbR\) such that
        \begin{itemize}
            \item \(\forall x,y \in \kk^\times, v(xy) = v(x) + v(y)\);
            \item \(\forall x,y \in \kk^\times, v(x + y) \geq \min\{v(x), v(y)\}\).
        \end{itemize}
        We can extend \(v\) to the whole field \(\kk\) by defining \(v(0) = +\infty\).
        Fix a real number \(\varepsilon \in (0,1)\).
        Then \(v\) induces an absolute value \(|\cdot|_v: \kk \to \bbR_+\) defined by \(|x|_v = \varepsilon^{v(x)}\) for each \(x \in \kk\).

        In some literature, the valuation \(v\) is called an \emph{valuation} and the induced absolute value \(|\cdot|_v\) is called a \emph{multiplicative valuation}.
        In this note, the term \emph{valuation} always refers to the multiplicative valuation (i.e., absolute value).
    \end{remark}

    \begin{example}\label{eg:trivial_absolute_value_on_any_fields}
        Let \(\kk\) be a field.
        The \emph{trivial absolute value} on \(\kk\) is defined as
        \[ \|x\| := \begin{cases}
            0, & x = 0; \\
            1, & x \neq 0.
        \end{cases} \]
    \end{example}

    \begin{definition}\label{def:archimedean_and_non-archimedean_place}
        An absolute value \(|\cdot|\) on a field \(\kk\) is called \emph{non-archimedean} if it satisfies the strong triangle inequality
        \[
            |x + y| \leq \max\{|x|, |y|\} \quad \text{for all } x, y \in \kk.
        \]
        Otherwise, it is called \emph{archimedean}.
    \end{definition}

    \begin{proposition}\label{prop:archimedean_place_iff_Z_is_unbounded}
        
    \end{proposition}

    \begin{notation}\label{notation:notation_for_valuations_on_fields}
        Let \(\kk\) be a field.
        We denote by \(M_\kk\) the set of all absolute values (i.e., valuations) on \(\kk\).
        For each \(v \in M_\kk\), we also call \(v\) a \emph{place} of \(\kk\).
    \end{notation}

\subsection{Non-archimedean place}


\subsection{Number field case}

    In this section, let \(\kk\) be a number field.

    \begin{theorem}\label{thm:place_of_number_fields}
        Let \(\kk\) be a number field.
        Then 
        \[ M_\kk^\infty = \{\text{embeddings } \sigma: \kk \to \bbC\} \]
        and
        \[ M_\kk^\rmf = \{\text{non-zero prime ideals } \frakp \subseteq \calO_\kk\}. \]
        \Yang{To be revised.}
    \end{theorem}

    \begin{proposition}[Product formula]\label{prop:product_formula_for_number_fields}
        Let \(\kk\) be a number field.
        For each \(x \in \kk^\times\), we have
        \[
            \prod_{v \in M_\kk} |x|_v^{n_v} = 1,
        \]
        where
        \[
            n_v := \begin{cases}
                [\kk_v : \bbR], & v \in M_\kk^\infty; \\
                1, & v \in M_\kk^0.
            \end{cases}
        \]
        \Yang{To be revised.}
    \end{proposition}