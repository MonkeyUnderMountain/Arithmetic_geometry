\section{Affinoid algebras}

% \subsection{The first properties}

    \begin{definition}\label{def:affinoid_algebras}
        Let \(R\) be a non-archimedean banach ring.
        A banach \(R\)-algebra \(A\) is called a \emph{\(R\)-affinoid algebra} if there exists an admissible surjective homomorphism
        \[
            \varphi: R\{ \underline{T/r} \} \twoheadrightarrow A
        \]
        for some \(r = (r_1, \ldots, r_n) \in \bbR_{>0}^n\).

        If one can choose \(r_1 = \cdots = r_n = 1\), then we say that \(A\) is a \emph{strictly \(R\)-affinoid algebra}.
    \end{definition}

    \begin{example}\label{eg:affinoid_algebra_for_trivial_norm}
        Suppose that the norm on \(R\) is trivial.
        Then a strictly \(R\)-affinoid algebra is just an \(R\)-algebra of finite type equipped with the trivial norm.
    \end{example}

    \begin{example}\label{eg:normed_localization_as_affinoid_algebra}
        Let \(A\) be an \(R\)-affinoid algebra, \(f_i,g_j \in A\) and \(r_i,s_j \in \bbR_{>0}\).
        Define
        \begin{align*}
            A\left\{ \underline{f/r} \right\} & := A\left\{ \underline{T/r} \right\} / (T_i - f_i), \\
            A\left\{ \underline{f/r}, \underline{s/g} \right\} & := A\left\{ \underline{T/r}, \underline{S/s} \right\} / (T_i - f_i, g_j S_j - 1). 
        \end{align*}
        Suppose that \(g \in R\) with \((f_i,g) = A\).
        Then we can define
        \[
            A\left\{ \underline{(f/g)/r} \right\} := A\left\{ \underline{T/r} \right\} / (g T_i - f_i).
        \]
        All of the above three algebras are again \(R\)-affinoid algebras.
        They can be viewed as normed analogues of localizations.
        \Yang{To be replaced.}
    \end{example}

    In the rest of this section, we fix a complete non-archimedean field \(\kk\) and consider \(\kk\)-affinoid algebras.


\subsection{Strict case}

    First we consider strictly \(\kk\)-affinoid algebras, which are well-studied in the era of rigid analytic geometry.

    \begin{proposition}\label{prop:automorphism_of_T_n_over_k}
        Let \(\kk\) be a complete non-archimedean field and \(f \in \kk\{\underline{T}\}\).
        Then there exists an automorphism \(\varphi\) of \(\kk\{\underline{T}\}\) over \(\kk\) such that \(\varphi(f)\) is \(T_n\)-distinguished, 
        i.e., \(\varphi(f) \in A\{T_n\}\) is distinguished in the variable \(T_n\) with \(A = \kk\{T_1, \ldots, T_{n-1}\}\).
    \end{proposition}
    \begin{proof}
        \Yang{To be added.}
    \end{proof}

    \begin{proposition}\label{prop:strictly_Tate_algebra_is_noetherian_factorial_and_Jacobson}
        Let \(\kk\) be a complete non-archimedean field.
        Then the Tate algebra \(\kk\{\underline{T}\}\) is noetherian, factorial, and Jacobson.
    \end{proposition}
    \begin{proof}
        \Yang{To be completed.}
    \end{proof}

    \begin{corollary}\label{prop:affinoid_algebra_is_noetherian}
        Strictly \(\kk\)-affinoid algebras are noetherian.
    \end{corollary}
    \begin{proof}
        \Yang{To be completed.}
    \end{proof}


    
    \begin{theorem}\label{thm:noetherian_normalization_theorem}
        Let \(A\) be a strictly \(\kk\)-affinoid algebra.
        Then there exists a finite injective admissible homomorphism
        \[
            \varphi: \kk\{ T_1, \ldots, T_d \} \hookrightarrow A.
        \]
    \end{theorem}
    \begin{proof}
        \Yang{To be completed.}
    \end{proof}



    \begin{proposition}\label{prop:the_norm_on_affinoid_is_bounded_by_spectral_radius}
        Let \(A\) be an \(\kk\)-affinoid algebra.
        Then there exists a constant \(C > 0\) and \(N > 0\) such that for all \(f \in A\) and \(n \geq N\), we have
        \[
            \|f^n\| \leq C \rho(f)^n.   
        \]
        In particular, \(\Qnil(A) = \nil(A)\).

        Furthermore, if \(A\) is reduced, we have 
        \[
            \|f\| \leq C \rho(f)
        \]
        for all \(f \in A\).
    \end{proposition}
    \begin{proof}
        \Yang{To be completed.}
    \end{proof}



\subsection{General case}


    \begin{proposition}\label{prop:affinoid_algebra_is_strict_when_radius_in_the_radical_of_valuation_set}
        Let \(A\) be an affinoid \(\kk\)-algebra.
        If and only if \(\rho(f) \in \sqrt{|\kk|} \cup \{0\}\) for all \(f \in A\), then \(A\) is strict.
        \Yang{To be complete.}
    \end{proposition}
    \begin{proof}
        \Yang{To be completed.}
    \end{proof}
    
    \begin{definition}\label{def:restricted_Laurent_series}
        Let \(\kk\) be a non-archimedean field.
        We define the \emph{ring of restricted Laurent series} over \(\kk\) as 
        \[ \KK_r = \bfL_{\kk,r} := \kk\{T/r,r/T\}. \]
    \end{definition}

    \Yang{Is \(\KK_r\) always a field?}
    \Yang{Do we have \(\bfL_{\kk,r} = \Frac (\kk\{T/r\})\)?}

    \begin{proposition}\label{prop:restricted_Laurent_series_is_a_field_when_r_is_not_root_of_absolute_value}
        Let \(\kk\) be a non-archimedean field.
        If \(r \notin \sqrt{|\kk^\times|}\), then \(\KK_r\) is a complete non-archimedean field with non-trivial absolute value extending that of \(\kk\).
    \end{proposition}

    \Yang{Tensor with \(\bfK_r\).}

    \begin{proposition}\label{prop:tensor_K_r_making_A_strictly}
        Let \(A\) be a \(\kk\)-affinoid algebra.
        Then there exists \(r_i \in \bbR_{>0}\) such that
        \[ \KK_{\underline{r}} \widehat{\otimes}_\kk A \]
        is a strictly \(\KK_{\underline{r}}\)-affinoid algebra.
    \end{proposition}

% \subsection{Noetherian normalization theorem}




% \subsection{Finite modules over affinoid algebras}

    There are three different categories of finite modules over an affinoid algebra \(A\):
    \begin{itemize}
        \item The category \(\Banmod_A\) of finite banach \(A\)-modules with \(A\)-linear maps as morphisms.
        \item The category \(\Banmod_A^b\) of finite banach \(A\)-modules with bounded \(A\)-linear maps as morphisms.
        \item The category \(\module_A\) of finite \(A\)-modules with all \(A\)-linear maps as morphisms.
    \end{itemize}

    \begin{theorem}\label{thm:equivalent_of_categroy_of_finite_banach_module_and_algebraic_module}
        Let \(A\) be an affinoid \(\kk\)-algebra.
        Then the category of finite banach \(A\)-modules with bounded \(A\)-linear maps as morphisms is equivalent to the category of finite \(A\)-modules with \(A\)-linear maps as morphisms.
        \Yang{To be revised.}
    \end{theorem}

    For simplicity, we will just write \(\mod_A\) to denote the category of finite banach \(A\)-modules with bounded \(A\)-linear maps as morphisms.


% \subsection{Tate algebras and Weierstrass division}

%     \begin{definition}\label{def:distinguished_degree_of_tate_algebra}
%         Let \(R\) be a non-archimedean banach ring and \(r \in \bbR_{>0}\).
%         A restricted power series \(f = \sum_{\alpha \in \bbN^n} a_\alpha T^\alpha \in R\{ \underline{T/r} \} \) is said to be \emph{distinguished in the variable \(T_n\) of degree \(d\)} if
%         \begin{itemize}
%             \item \(a_{\alpha} \in R\) is a unit for \(\alpha = (0, \ldots, 0, d)\);
%             \item \(\|a_\alpha\| r^\alpha < \|a_{(0,\ldots,0,d)}\| r_n^d\) for all \(\alpha_n < d\).
%         \end{itemize}
%         \Yang{To be revised.}
%     \end{definition}

%     \begin{proposition}\label{prop:restricted_power_series_invertible_iff_the_constant_item_is_invertible_and_controls_others}
%         Let \(R\) be a non-archimedean banach ring.
%         An element \(f = \sum_{\alpha \in \bbN^n} a_\alpha T^\alpha \in R\{ \underline{T/r} \} \) is invertible if and only if \(a_0\) is invertible in \(R\) and \(\|a_0\| > \|a_\alpha\| r^\alpha\) for all \(\alpha \neq 0\).
%         % \Yang{To be checked.}
%     \end{proposition}
%     \begin{proof}
%         Multiplying by \(a_0^{-1}\), we can reduce to the case \(a_0 = 1\).
%         Let \(g = \sum_{\alpha \in \bbN^n} b_\alpha T^\alpha\) be the inverse of \(f\) in \(R[[\underline{T}]]\).
%         Then we have
%         \[ f \cdot g = \sum_{\alpha \in \bbN^n} a_\alpha T^\alpha \cdot \sum_{\beta \in \bbN^n} b_\beta T^\beta = \sum_{\gamma \in \bbN^n} \left( \sum_{\alpha + \beta = \gamma} a_\alpha b_\beta \right) T^\gamma = 1. \]
%         That is, for every \(\gamma \neq 0 \in \bbN^n\),
%         \[ b_{\gamma} = - \sum_{\substack{\alpha + \beta = \gamma \\ \alpha \neq 0}} a_\alpha b_\beta. \]
%         Let \(A = \|f-1\| < 1\).
%         We show that for every \(m \in \bbN\), there exists \(C_m > 0\) such that for all \(\alpha \in \bbN^n\) with \(|\alpha| \geq C_m\), we have \(\|b_\alpha\| r^\alpha \leq A^m\).
%         For \(m = 0\), note that \(b_0 = 1\).
%         By induction on \(\gamma\) with respect to the total order \(\leq_{\text{total}}\), we have
%         \[ \|b_\gamma\| r^\gamma \leq \max_{\substack{\alpha + \beta = \gamma \\ \alpha \neq 0}} \|a_\alpha\| r^\alpha \cdot \|b_\beta\| r^\beta \leq A \max_{\beta <_{\text{total}} \gamma} \|b_\beta\| r^\beta \leq 1. \]
%         Suppose that the claim holds for \(m\).
%         There exists \(D_{m+1} \in \bbN\) such that for all \(\alpha \in \bbN^n\) with \(|\alpha| \geq D_{m+1}\), we have \(\|a_\alpha\| r^\alpha \leq A^{m+1}\).
%         Set \(C_{m+1} = C_m + D_{m+1} + 1\).
%         For any \(\gamma \in \bbN^n\) with \(|\gamma| \geq C_{m+1}\), we have
%         \[ \|b_\gamma\| r^\gamma \leq \max_{\substack{\alpha + \beta = \gamma \\ \alpha \neq 0}} \|a_\alpha\| r^\alpha \cdot \|b_\beta\| r^\beta \leq \max \left\{ A^{m+1}, A \cdot A^m \right\} = A^{m+1} \]
%         since either \(|\alpha| \geq D_{m+1}\) or \(|\beta| \geq C_m\).
%         Thus by induction, we have \(\|b_\alpha\| r^\alpha \to 0\) as \(|\alpha| \to +\infty\).
%         It follows that \(g \in R\{ \underline{T/r} \} \).
%     \end{proof}

%     \begin{theorem}[Weierstrass preparation theorem]\label{thm:Weierstrass_preparation_theorem_Tate_algebra}
%         Let \(\kk\) be a complete non-archimedean field.
%         Let \(f \in \kk\{ \underline{T/r} \} \) be a restricted power series that is distinguished in the variable \(T_n\) of degree \(d\), i.e.,
%         \[ f = \sum_{\alpha \in \bbN^{n-1}} a_\alpha T^\alpha + \sum_{\alpha_n \geq 1} a_\alpha T^\alpha \]
%         with \(a_{(0,\ldots,0,d)}\) being a unit in \(\kk\{ \underline{T/r} \} \) and \(\|a_\alpha\| r^\alpha < \|a_{(0,\ldots,0,d)}\| r_n^d\) for all \(\alpha_n < d\).
%         Then there exists a unique monic polynomial \(P \in \kk\{ \underline{T/r} \} [T_n]\) of degree \(d\) in \(T_n\) and a unique unit \(U \in \kk\{ \underline{T/r} \} \) such that
%         \[ f = P \cdot U. \]
%         \Yang{To be checked.}
%     \end{theorem}

%     \begin{theorem}[Weierstrass division theorem]\label{thm:Weierstrass_division_theorem_Tate_algebra}
%         Let \(\kk\) be a complete non-archimedean field.
%         Let \(f \in \kk\{ \underline{T/r} \} \) be a restricted power series that is distinguished in the variable \(T_n\) of degree \(d\).
%         Then for every \(g \in \kk\{ \underline{T/r} \} \), there exists a unique \(Q \in \kk\{ \underline{T/r} \} \) and a unique polynomial \(R \in \kk\{ \underline{T/r} \} [T_n]\) of degree less than \(d\) in \(T_n\) such that
%         \[ g = Q \cdot f + R. \]
%         \Yang{To be checked.}
%     \end{theorem}

%     \begin{proposition}\label{prop:algebraic_spectra_of_Tate_algebra}
%         Let \(\kk\) be a complete non-archimedean field and \(r = (r_1, \ldots, r_n) \in \bbR_+^n\).
%         Then 
%         \[ \Spec \kk\{ \underline{T/r} \} = \{  \}, \]
%         where
%     \end{proposition}


