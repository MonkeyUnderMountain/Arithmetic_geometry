\section{Convergent and restricted power series}


     \begin{notation}\label{notation:mult-label_for_Tate_algebra}
        Let \(T = (T_1, \ldots, T_n)\) be a tuple of \(n\) indeterminates, \(r = (r_1, \ldots, r_n)\) be a tuple of \(n\) positive real numbers, and \(\alpha = (\alpha_1, \ldots, \alpha_n) \in \bbN^n\) be a multi-index.
        We use the following notations:
        \begin{itemize}
            \item \(T^\alpha := T_1^{\alpha_1} T_2^{\alpha_2} \cdots T_n^{\alpha_n}\) and \(r^\alpha := r_1^{\alpha_1} r_2^{\alpha_2} \cdots r_n^{\alpha_n}\);
            \item \(\underline{T/r} := (T_1/r_1, T_2/r_2, \ldots, T_n/r_n)\);
            \item \(|\alpha| := \alpha_1 + \alpha_2 + \cdots + \alpha_n\);
            \item \(\alpha \leq_{\text{total}} \beta\) if and only if for all \(i = 1, \ldots, n\), we have \(\alpha_i \leq \beta_i\);
            % \item \(E(x,\underline{r}) = \{y \in \kk^n \mid \|y_i - x_i\| \leq r_i, i = 1, \ldots, n\}\) and \(B(x,\underline{r}) = \{y \in \kk^n \mid \|y_i - x_i\| < r_i, i = 1, \ldots, n\}\) for \(x = (x_1, \ldots, x_n) \in \kk^n\);
            \item Let \(\{x_{\alpha}\}_{\alpha \in \bbN^n}\) be a set of elements in a metric space \(X\) indexed by multi-indices \(\alpha \in \bbN^n\).
                We say that \(\lim_{|\alpha| \to +\infty} x_\alpha = x \in X\) if for every \(\varepsilon > 0\), there exists \(N \in \bbN\) such that for all \(\alpha \in \bbN^n\) with \(|\alpha| > N\), we have \(d(x_\alpha, x) < \varepsilon\).
        \end{itemize}
    \end{notation}


\subsection{Absolutely convergent power series}

    \begin{definition}\label{def:ring_of_absolutely_convergent_power_series}
        Let \(R\) be a banach ring and \(r > 0\) be a real number.
        We define the \emph{ring of absolutely convergent power series} over \(\kk\) with radius \(r\) as
        \[ R\left<T/r\right> \coloneqq \left\{\sum_{n=0}^{\infty} a_n T^n \in R[[T]] : \sum_{n=0}^{\infty} \|a_n\| r^n < \infty \right\}. \]
        Equipped with the norm \(\|\sum_{n=0}^{\infty} a_n T^n\| := \sum_{n=0}^{\infty} \|a_n\| r^n\), the ring \(R\left<T/r\right>\) is a banach ring.

        For a tuple of \(n\) indeterminates \(T = (T_1, \ldots, T_n)\) and a tuple of \(n\) positive real numbers \(r = (r_1, \ldots, r_n)\), we define
        \[ R\left<\underline{T/r}\right> \coloneqq R\left<T_1/r_1, \ldots, T_n/r_n\right> := R\left<T_1/r, \ldots, T_{n-1}/r_{n-1}\right>\left<T_n/r_n\right>. \]
    \end{definition}

    Note that if \(R\) has trivial norm, then
    \[ R\left<T/r\right> = \begin{cases}
        R[[T]], & \text{if } r < 1; \\
        R[T], & \text{if } r \geq 1.
    \end{cases} \]

    \Yang{To add the spectral of absolutely convergent power series.}


\subsection{Tate algebras}

    \begin{definition}\label{def:Tate_algebra_over_banach_ring}
        Let \(R\) be a non-archimedean banach ring.
        Let \(T = (T_1, \ldots, T_n)\) be a tuple of \(n\) indeterminates and \(r = (r_1, \ldots, r_n)\) be a tuple of \(n\) positive real numbers.
        The \emph{Tate algebra} (or \emph{ring of restricted power series}) is defined as 
        \[
            R\langle \underline{r^{-1}T} \rangle := R \{ \underline{r^{-1}T} \} := \left\{ \sum_{\alpha \in \bbN^n} a_\alpha T^\alpha \mid a_\alpha \in R, \lim_{|\alpha| \to +\infty} \|a_\alpha\| r^\alpha = 0 \right\}.
        \]
    \end{definition}

    \begin{proposition}\label{prop:Tate_algebra_is_a_banach_algebra_over_R}
        Let \(R\) be a non-archimedean banach ring.
        Then the Tate algebra \(R\{ \underline{T/r} \} \) is a non-archimedean multiplicative banach \(R\)-algebra with respect to the \emph{gauss norm}
        \[
            \left\| \sum_{\alpha \in \bbN^n} a_\alpha T^\alpha \right\| := \sup_{\alpha \in \bbN^n} \|a_\alpha\|r^\alpha = \max_{\alpha \in \bbN^n} \|a_\alpha\|r^\alpha.
        \]
    \end{proposition}
    \begin{proof}
        The proof splits into several parts.
        Every parts is straightforward and standard.

        \begin{step}\label{step_in_prop:Tate_algebra_is_a_banach_algebra_over_R:R-algebra}
            We first show that \(R\{ \underline{T/r} \} \) is a \(R\)-algebra.
        \end{step}

        Easily to see that it is closed under addition and scalar multiplication.
        Suppose that \(f = \sum_{\alpha \in \bbN^n} a_\alpha T^\alpha\) and \(g = \sum_{\alpha \in \bbN^n} b_\alpha T^\alpha\) are two nonzero elements in \(R\{ \underline{T/r} \} \).
        Given \(\varepsilon > 0\), there exists \(N \in \bbN\) such that for all \(|\alpha| > N\), we have \(\|a_\alpha\| r^\alpha < \varepsilon/\|g\|\) and \(\|b_\alpha\| r^\alpha < \varepsilon/\|f\|\).
        For any \(|\gamma| > 2N\), we have
        \[
            \left\| \sum_{\alpha + \beta = \gamma} a_\alpha b_\beta \right\| r^\gamma \leq \max_{\alpha + \beta = \gamma} \|a_\alpha\| r^\alpha \cdot \|b_\beta\| r^\beta < \max\left\{ \frac{\varepsilon}{\|g\|} \|b_{\beta}\|r^\beta, \frac{\varepsilon}{\|f\|} \|a_{\alpha}\|r^\alpha \right\} \leq \varepsilon.
        \]
        Hence \(f \cdot g \in R\{ \underline{T/r} \} \) and it shows that \(R\{ \underline{T/r} \} \) is a \(R\)-algebra.

        \begin{step}\label{step_in_prop:Tate_algebra_is_a_banach_algebra_over_R:normed_R-algebra}
            Show that the gauss norm is a non-archimedean norm on \(R\{ \underline{T/r} \} \).
        \end{step}

        The linearity and positive-definiteness of the gauss norm are direct from the definition.
        We have
        \[
            \|f + g\| = \sup_{\alpha \in \bbN^n} \|a_\alpha + b_\alpha\| r^\alpha \leq \sup_{\alpha \in \bbN^n} \max\{\|a_\alpha\| + \|b_\alpha\|\} r^\alpha \leq \max\{\|f\|, \|g\|\}
        \]
        and 
        \begin{align*}
            \| f \cdot g \| &= \left\| \sum_{\gamma \in \bbN^n} \left( \sum_{\alpha + \beta = \gamma} a_\alpha b_\beta \right) T^\gamma \right\| = \sup_{\gamma \in \bbN^n} \left\| \sum_{\alpha + \beta = \gamma} a_\alpha b_\beta \right\| r^\gamma \\
            &\leq \sup_{\gamma \in \bbN^n} \max_{\alpha + \beta = \gamma} \|a_\alpha\| \|b_\beta\| r^\alpha r^\beta = \|a_{\alpha_0}\| r^{\alpha_0} \cdot \|b_{\beta_0}\| r^{\beta_0} \leq \|f\| \cdot \|g\|.
        \end{align*}
        These show that Tate algebra with the gauss norm is a non-archimedean normed \(\kk\)-algebra.

        \begin{step}\label{step_in_prop:Tate_algebra_is_a_banach_algebra_over_k:multiplicativity}
            Show that the gauss norm is multiplicative.
        \end{step}

        Suppose that \(\|f\| = \|a_{\alpha_1}\| r^{\alpha_1}\) and \(\|a_{\alpha}\|r^\alpha < \|f\|\) for all \(\alpha <_{\text{total}} \alpha_1\).
        Similar to \(\|b_{\beta_1}\| r^{\beta_1}\).
        Then we have
        \[
            \|f\| \cdot \|g\| = \|a_{\alpha_1}\| r^{\alpha_1} \cdot \|b_{\beta_1}\| r^{\beta_1} = \max_{\alpha + \beta = \alpha_1 + \beta_1} \|a_\alpha\| \|b_\beta\| r^\alpha r^\beta = \left\| \sum_{\alpha + \beta = \alpha_1 + \beta_1} a_\alpha b_\beta \right\| r^{\alpha_1 + \beta_1} \leq \| f \cdot g \|,
        \]
        where the third equality holds since \((\alpha_1, \beta_1)\) is the unique pair such that \(\|a_{\alpha_1}\| r^{\alpha_1} \cdot \|b_{\beta_1}\| r^{\beta_1}\) is maximized and by \cref{prop:all_triangles_in_ultra-metric_space_are_isosceles}.
        Thus the gauss norm is multiplicative.

        \begin{step}\label{step_in_prop:Tate_algebra_is_a_banach_algebra_over_R:completeness}
            Finally show that \(R\{ \underline{T/r} \} \) is complete with respect to the gauss norm.
        \end{step}

        Let \(\{f_m = \sum a_{\alpha,m}T^\alpha\}\) be a cauchy sequence in \(R\{ \underline{T/r} \} \).
        We have
        \[ \|a_{\alpha,m} - a_{\alpha,l}\| r^\alpha \leq \|f_m - f_l\|. \]
        Thus for each \(\alpha \in \bbN^n\), the sequence \(\{a_{\alpha,m}\}\) is a cauchy sequence in \(R\).
        Since \(R\) is complete, set \(a_\alpha := \lim_{m \to +\infty} a_{\alpha,m}\) and \(f = \sum_{\alpha \in \bbN^n} a_\alpha T^\alpha\).
        Given \(\varepsilon > 0\), there exists \(M \in \bbN\) such that for all \(m,l > M\), we have \(\|f_m - f_l\| < \varepsilon\).
        Fixing \(m > M\), there exists \(N \in \bbN\) such that for all \(|\alpha| > N\), we have \(\|a_{\alpha,m}\| r^\alpha < \varepsilon\).
        Hence for all \(|\alpha| > N\) and \(l > M\), we have
        \[ \|a_{\alpha,l}\| r^\alpha \leq \|a_{\alpha,l} - a_{\alpha,m}\| r^\alpha + \|a_{\alpha,m}\| r^\alpha < 2\varepsilon. \]
        Taking \(l \to +\infty\), we have \(\|a_\alpha\| r^\alpha \leq 2\varepsilon\) for all \(|\alpha| > N\).
        It follows that \(f \in \kk\{ \underline{T/r} \} \).

        For every \(\varepsilon > 0\), there exists \(N \in \bbN\) such that for all \(m,l > N\), we have \(\|f_m - f_l\| < \varepsilon\).
        Thus for all \(\alpha \in \bbN^n\) and \(m,l > N\), we have
        \[ \|a_{\alpha,m} - a_{\alpha,l}\| r^\alpha \leq \|f_m - f_l\| < \varepsilon. \]
        Taking \(l \to +\infty\), we have \(\|a_{\alpha,m} - a_\alpha\| r^\alpha \leq \varepsilon\) for all \(m > N\).
        It follows that
        \[ \|f - f_m\| = \sup_{\alpha \in \bbN^n} \|a_\alpha - a_{\alpha,m}\| r^\alpha \leq \varepsilon \]
        for all \(m > N\).
        \Yang{To be revised, the original version is for a field.}
    \end{proof}

    % \subsection{Reduction}

    \begin{example}\label{eg:reduction_ring_of_standard_tate_algebra_over_a_banach_ring}
        Let \(R\) be a non-archimedean banach ring and \(A = R\{T\}\) be the Tate algebra in one variable over \(R\).
        %  field and \(A = \kk\{ \underline{T} \}\) be a Tate algebra over \(\kk\).
        Then we have
        \[ A^\circ = \left\{ \sum_{n\geq 0} a_n T^n : |a_n| \leq 1 \text{ for all } n \in \bbN \right\}  = R^\circ \{ T \}, \]
        and
        \[ A^{\circ \circ} = \left\{ \sum_{n\geq 0} a_n T^n : |a_n| < 1 \text{ for all } n \in \bbN \right\}  = R^{\circ \circ} \{ T \}. \]
        Since the norm of items in a restricted power series will tend to \(0\), we have
        \[ \widetilde{A} = \widetilde{R} [\underline{T}]. \] 
    \end{example}

    \begin{example}\label{eg:reduction_ring_of_general_tate_algebra_over_a_banach_ring}
        Let \(R\) is a multiplicative non-archimedean banach ring.
        Set 
        \[\sqrt{|R|^{-1}} = \{ r \in \bbR_{>0} : r^{-n} \in |R| \text{ for some } n \in \bbN_{>0} \}.\]
        Fix \(r \in \bbR_{>0}^n\), consider the Tate algebra \(A = R\{T/r\}\).
        
        Suppose that \(r \in \sqrt{|R|^{-1}}\).
        Let \(n\) be the minimal positive integer such that \(r^n \in |R|^{-1}\) and 
        \[ \widetilde{M}_k:=\{a \in R: |a| = r^{-nk}\}/\{a \in R: |a| < r^{-nk}\}. \]
        For \(a_m T^m\) with \(n \not\mid m\), we have\( \|a_m T^m\| = |a_m| r^m \leq 1 \implies |a_m|r^m < 1\).
        Hence 
        \[ \widetilde{R\{T/r\}} = \widetilde{R} \oplus \widetilde{M}_1 T^n \oplus \widetilde{M}_2 T^{2n} \oplus \widetilde{M}_3 T^{3n} \oplus \cdots. \]
        In case \(R = \kk\) is a non-archimedean field, we have \(\widetilde{M}_k \cong \widetilde{\kk}\) by choosing an element \(c \in \kk\) with \(|c| = r^{-n}\).
        Hence 
        \[ \widetilde{\kk\{T/r\}} \cong \calk_\kk[T^n]. \]

        Suppose that \(r \notin \sqrt{|R|^{-1}}\).
        Then for every \(\|a_n T^n\| = a_n r^n \leq 1\), we have \(|a_n| < 1\).
        It follows that
        \[ \widetilde{R\{T/r\}} = \widetilde{R}. \]
        % \Yang{To be continued.}
    \end{example}


\subsection{Weierstrass preparation}

    \begin{definition}\label{def:degree_of_restricted_power_series}
        Let \(R\) be a non-archimedean banach ring and \(A = R\{T/r\}\).
        For \(f = \sum_{a_n \in \bbN} a_n T^n \in A\), we define the \emph{degree} of \(f\) as 
        \[ \deg f := \max \{ n \in \bbN : \|a_n\| r^n = \|f\| \}. \]
    \end{definition}

    It is interesting to note that if \(R\) has trivial norm, then \(\deg f\) coincides with the usual degree of polynomials when \(r \geq 1\) and the order of formal power series when \(r < 1\).

    \begin{definition}\label{def:distinguished_restricted_power_series}
        Let \(R\) be a non-archimedean banach ring and \(A = R\{T/r\}\).
        A restricted power series \(f = \sum_{n \in \bbN} a_n T^n \in A\) of degree \(d\) is said to be \emph{distinguished} if \(a_d\) is invertible in \(R\).
    \end{definition}

    \begin{proposition}\label{prop:restricted_power_series_invertible_iff_the_constant_item_is_invertible_and_controls_others}
        Let \(R\) be a non-archimedean banach ring.
        An element \(f\) is invertible if and only if \(\deg f = 0\) and the constant item of \(f\) is invertible in \(R\).
        % \Yang{To be checked.}
    \end{proposition}
    \begin{proof}
        Multiplying by \(a_0^{-1}\), we can reduce to the case \(a_0 = 1\).
        Let \(g = \sum_{\alpha \in \bbN^n} b_\alpha T^\alpha\) be the inverse of \(f\) in \(R[[\underline{T}]]\).
        Then we have
        \[ f \cdot g = \sum_{\alpha \in \bbN^n} a_\alpha T^\alpha \cdot \sum_{\beta \in \bbN^n} b_\beta T^\beta = \sum_{\gamma \in \bbN^n} \left( \sum_{\alpha + \beta = \gamma} a_\alpha b_\beta \right) T^\gamma = 1. \]
        That is, for every \(\gamma \neq 0 \in \bbN^n\),
        \[ b_{\gamma} = - \sum_{\substack{\alpha + \beta = \gamma \\ \alpha \neq 0}} a_\alpha b_\beta. \]
        Let \(A = \|f-1\| < 1\).
        We show that for every \(m \in \bbN\), there exists \(C_m > 0\) such that for all \(\alpha \in \bbN^n\) with \(|\alpha| \geq C_m\), we have \(\|b_\alpha\| r^\alpha \leq A^m\).
        For \(m = 0\), note that \(b_0 = 1\).
        By induction on \(\gamma\) with respect to the total order \(\leq_{\text{total}}\), we have
        \[ \|b_\gamma\| r^\gamma \leq \max_{\substack{\alpha + \beta = \gamma \\ \alpha \neq 0}} \|a_\alpha\| r^\alpha \cdot \|b_\beta\| r^\beta \leq A \max_{\beta <_{\text{total}} \gamma} \|b_\beta\| r^\beta \leq 1. \]
        Suppose that the claim holds for \(m\).
        There exists \(D_{m+1} \in \bbN\) such that for all \(\alpha \in \bbN^n\) with \(|\alpha| \geq D_{m+1}\), we have \(\|a_\alpha\| r^\alpha \leq A^{m+1}\).
        Set \(C_{m+1} = C_m + D_{m+1} + 1\).
        For any \(\gamma \in \bbN^n\) with \(|\gamma| \geq C_{m+1}\), we have
        \[ \|b_\gamma\| r^\gamma \leq \max_{\substack{\alpha + \beta = \gamma \\ \alpha \neq 0}} \|a_\alpha\| r^\alpha \cdot \|b_\beta\| r^\beta \leq \max \left\{ A^{m+1}, A \cdot A^m \right\} = A^{m+1} \]
        since either \(|\alpha| \geq D_{m+1}\) or \(|\beta| \geq C_m\).
        Thus by induction, we have \(\|b_\alpha\| r^\alpha \to 0\) as \(|\alpha| \to +\infty\).
        It follows that \(g \in R\{ \underline{T/r} \} \).
        \Yang{To be revised.}
    \end{proof}

    \begin{proposition}\label{prop:Tate_alg_in_one_variable_over_field_is_ED}
        Let \(\kk\) be a complete non-archimedean field and \(r > 0\) be a positive real number.
        Then the Tate algebra \(\kk\{T/r\}\) is an euclidean domain with respect to the degree defined in \cref{def:degree_of_restricted_power_series}.
        \Yang{To be added.}
    \end{proposition}
    \begin{proof}
        Let \(f, g \in \kk\{T/r\}\) be two elements with \(g \neq 0\).
        Denote \(n = \deg f\) and \(m = \deg g\).
        We need to find \(q, r \in \kk\{T/r\}\) such that
        \[ f = q \cdot g + r, \quad \deg r < \deg g.\]
        \Yang{To be added.}
    \end{proof}

    \begin{definition}\label{def:Weierstrass_polynomial}
        Let \(R\) be a non-archimedean banach ring and \(A = R\{ T/r \} \).
        A Weierstrass polynomial is a monic polynomial \(P \in A[T] \subset R\{T/r\}\) whose two degrees as a polynomial and as a restricted power series coincide.
        % \Yang{To be checked.}
    \end{definition}

    \begin{theorem}[Weierstrass preparation theorem]\label{thm:Weierstrass_preparation_theorem_Tate_algebra}
        Let \(R\) be a non-archimedean banach ring.
        Let \(f \in R\{T/r\}\) be a distinguished restricted power series of degree \(d\).
        Then there exists a unique Weierstrass polynomial \(p \in R[T]\) of degree \(d\) and a unique unit \(u \in R\{T/r\}\) such that
        \[ f = p \cdot u. \]
        \Yang{To be checked.}
    \end{theorem}
    \begin{proof}
        \Yang{To be added.}
    \end{proof}

    % \begin{theorem}[Weierstrass division theorem]\label{thm:Weierstrass_division_theorem_Tate_algebra}
    %     Let \(R\) be a non-archimedean banach ring.
    %     Let \(f \in R\{T/r\}\) be a distinguished restricted power series of degree \(d\).
    %     Then for every \(g \in R\{T/r\}\), there exists a unique \(q \in R\{T/r\}\) and a unique polynomial \(p \in R[T]\) of degree less than \(d\) such that
    %     \[ g = q \cdot f + p. \]
    %     \Yang{To be checked.}
    % \end{theorem}
    % \begin{proof}
    %     \Yang{To be added.}
    % \end{proof}

    \begin{remark}\label{rmk:Weierstrass_preparation_in_complex_formal_and_Tate_settings}
        In my knowledge, there are at least three different versions of Weierstrass preparation theorem under different settings:
        \begin{itemize}
            \item The classical Weierstrass preparation in complex analysis;
            \item The Weierstrass preparation for formal power series over complete noetherian local rings;
            \item The Weierstrass preparation for Tate algebras over non-archimedean banach rings.
        \end{itemize}


        Let \((R,\frakm)\) be a complete noetherian local ring.
        Note that there is also a Weierstrass preparation theorem for formal power series over \(R\) stating that 
        for every formal power series \(f \in R[[T]]\) whose reduction \(\overline{f} \in (R/\frakm)[[T]]\) is of order \(d\), 
        there exists a unique monic polynomial \(p \in R[T]\) of degree \(d\) and a unique unit \(u \in R[[T]]\) such that
        \[ p \equiv T^d \mod \frakm, \quad f = p \cdot u. \]

        \Yang{To be continued.}
    \end{remark}


% \subsection{Algebraic properties of Tate algebras}

%     \begin{corollary}\label{cor:Tate_algebra_is_factorial_or_noetherian_if_coefficient_ring_is}
%         Let \(R\) be a non-archimedean banach ring and \(A = R\{T/r\}\).
%         If \(R\) is factorial (resp. noetherian), then so is \(A\).
%     \end{corollary}

    % \begin{proposition}\label{prop:algebraic_spectra_of_Tate_algebra}
    %     Let \(\kk\) be a complete non-archimedean field and \(r = (r_1, \ldots, r_n) \in \bbR_+^n\).
    %     Then 
    %     \[ \Spec \kk\{ \underline{T/r} \} = \{  \}, \]
    %     where
    % \end{proposition}

