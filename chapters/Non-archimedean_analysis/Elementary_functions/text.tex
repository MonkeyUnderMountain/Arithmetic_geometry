\section{Elementary functions}

\subsection{Elementary functions}

    \begin{lemma}\label{lem:p-valuation_of_factorial}
        Let \(p\) be a prime number and \(n \in \bbN\).
        We have \(v_p(n!) = \). 
    \end{lemma}

    \Yang{Exponential, logarithmic, and the interpolation functions.}

    Fix a prime number \(p\) in the following and consider \(\kk = \bbQ_p, \bbC_p\), or \(\Omega_p\).
    Let \(r_p := p^{-1/(p-1)}\).

    \begin{construction}\label{constr:exponential_and_logarithmic_functions}
        The \emph{exponential function} \(\exp: \kk \to \kk\) is defined by the power series
        \[ \exp(x) := \sum_{n=0}^{+\infty} \frac{x^n}{n!}. \]
        % The radius of convergence of \(\exp(x)\) is \(+\infty\) if \(p = 2\) and \(p^{-1/(p-1)}\) if \(p > 2\).
        The \emph{logarithmic function} \(\log: 1 + \kk^{\circ\circ} \to \kk\) is defined by the power series
        \[ \log(1+x) := \sum_{n=1}^{+\infty} (-1)^{n+1} \frac{x^n}{n}. \]
        % The radius of convergence of \(\log(1+x)\) is \(1\).

        % Moreover, for every \(x\) in the domain of convergence of \(\exp\) and every \(y\) in the domain of convergence of \(\log\), we have
        % \[ \log(\exp(x)) = x, \quad \exp(\log(y)) = y. \]
        \Yang{To be checked.}
    \end{construction}

    \begin{definition}\label{def:Malher_series}
        Let 
    \end{definition}

    \begin{theorem}\label{thm:}
        The series converges.
    \end{theorem}