\section{Elementary functions}

\subsection{Exponential and logarithmic functions}

    Fix a prime number \(p\) in the following and consider \(\kk\) being a complete non-archimedean field with \(|p| = p^{-1}\).
    Let \(r_p := p^{-1/(p-1)}\).

    \begin{construction}\label{constr:exponential_and_logarithmic_functions}
        The \emph{exponential function} \(\exp: \kk \to \kk\) is defined by the power series
        \[ \exp(x) := \sum_{n=0}^{+\infty} \frac{x^n}{n!}. \]
        The \emph{logarithmic function} \(\log: 1 + \kk^{\circ\circ} \to \kk\) is defined by the power series
        \[ \log(1+x) := \sum_{n=1}^{+\infty} (-1)^{n+1} \frac{x^n}{n}. \]
        \Yang{To be checked.}
    \end{construction}

    Recall the following useful lemma regarding the \(p\)-adic valuation of factorials.

    \begin{lemma}\label{lem:p-valuation_of_factorial}
        Let \(p\) be a prime number and \(n \in \bbN\).
        We have 
        \[ v_p(n!) = \sum_{k=1}^{+\infty} \left\lfloor \frac{n}{p^k} \right\rfloor. \]
    \end{lemma}
    \begin{proof}
        \Yang{To be added.}
    \end{proof}

    \Yang{Exponential, logarithmic, and the interpolation functions.}

    \begin{proposition}\label{prop:fundamental_properties_of_p-adic_exp_and_log}
        We have the following properties:
        \begin{enumerate}
            \item the exponential function \(\exp\) converges on the open disk \(B(0, r_p)\);
            \item the logarithmic function \(\log\) converges on the open disk \(B(1, 1)\);
            \item endow \(B(0, r_p)\) with the group structure induced by addition in \(\kk\) and \(B(1, r_p)\) with the group structure induced by multiplication in \(\kk\), then \(\exp: B(0, r_p) \to B(1, r_p)\) is a group isomorphism with inverse \(\log: B(1, r_p) \to B(0, r_p)\).
        \end{enumerate}
        \Yang{To be checked.}
    \end{proposition}
    \begin{proof}
        \Yang{To be added.}
    \end{proof}

    \begin{proposition}\label{prop:log_on_the_ball_B_11}
        The logarithmic function \(\log\) defines a group homomorphism \(1 + \kk^{\circ\circ} \to \kk\) with kernel the group \(\mu_{p^\infty}\) of all \(p\)-power roots of unity.
        \Yang{To be checked.}
    \end{proposition}

    \Yang{continuation of exponential and logarithmic}

\subsection{Mahler series}

    \begin{notation}\label{notation:binomial_polynomial}
        We use \(\binom{x}{n}\) to denote the \emph{binomial polynomial} defined by
        \[ \binom{x}{n} := \frac{x(x-1)(x-2) \cdots (x-n+1)}{n!}. \]
    \end{notation}

    \begin{definition}\label{def:Mahler_series}

    \end{definition}

    \begin{theorem}\label{thm:}
        The series converges.
    \end{theorem}