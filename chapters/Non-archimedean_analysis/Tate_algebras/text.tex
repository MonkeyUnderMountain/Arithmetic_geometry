\section{Tate algebras}

 \begin{notation}\label{notation:mult-label_for_Tate_algebra}
        Let \(T = (T_1, \ldots, T_n)\) be a tuple of \(n\) indeterminates, \(r = (r_1, \ldots, r_n)\) be a tuple of \(n\) positive real numbers, and \(\alpha = (\alpha_1, \ldots, \alpha_n) \in \bbN^n\) be a multi-index.
        We use the following notations:
        \begin{itemize}
            \item \(T^\alpha := T_1^{\alpha_1} T_2^{\alpha_2} \cdots T_n^{\alpha_n}\) and \(r^\alpha := r_1^{\alpha_1} r_2^{\alpha_2} \cdots r_n^{\alpha_n}\);
            \item \(\underline{T/r} := (T_1/r_1, T_2/r_2, \ldots, T_n/r_n)\);
            \item \(|\alpha| := \alpha_1 + \alpha_2 + \cdots + \alpha_n\);
            \item \(\alpha \leq_{\text{total}} \beta\) if and only if for all \(i = 1, \ldots, n\), we have \(\alpha_i \leq \beta_i\);
            \item \(E(x,\underline{r}) = \{y \in \kk^n \mid \|y_i - x_i\| \leq r_i, i = 1, \ldots, n\}\) and \(B(x,\underline{r}) = \{y \in \kk^n \mid \|y_i - x_i\| < r_i, i = 1, \ldots, n\}\) for \(x = (x_1, \ldots, x_n) \in \kk^n\);
            \item Let \(\{x_{\alpha}\}_{\alpha \in \bbN^n}\) be a set of elements in a metric space \(X\) indexed by multi-indices \(\alpha \in \bbN^n\).
                We say that \(\lim_{|\alpha| \to +\infty} x_\alpha = x \in X\) if for every \(\varepsilon > 0\), there exists \(N \in \bbN\) such that for all \(\alpha \in \bbN^n\) with \(|\alpha| > N\), we have \(d(x_\alpha, x) < \varepsilon\).
        \end{itemize}
    \end{notation}


    % \begin{proposition}\label{prop:convergent_radius_of_power_series}
    %     Let \(\kk\) be a complete non-archimedean field and \(f = \sum_{\alpha \in \bbN^n} a_\alpha T^\alpha \in \kk[[T_1, \ldots, T_n]]\).
    %     \Yang{To be continued.}
    % \end{proposition}

    % \begin{proposition}
    %     % \label{prop:convergent_radius_of_power_series}
    %     Let \(\kk\) be a complete non-archimedean field and \(f = \sum_{n=0}^{+\infty} a_n T^n \in \kk[[T]]\).
    %     Set 
    %     \[
    %         R := \frac{1}{\limsup_{n \to +\infty} \|a_n\|^{1/n}} \in \bbR_{\geq 0} \cup \{+\infty\}.
    %     \]
    %     Then we have 
    %     \begin{enumerate}
    %         \item if \(R = 0\), then the series \(f(x)\) converges only at \(x = 0\);
    %         \item if \(R = +\infty\), then the series \(f(x)\) converges for all \(x \in \kk\);
    %         \item if \(0 < R < +\infty\), then the series \(f(x)\) converges for all \(x \in \kk\) with \(\|x\| < R\) and diverges for all \(x \in \kk\) with \(\|x\| > R\).
    %     \end{enumerate}
    %     Suppose that \(0 < R < +\infty\).
    %     Then the series \(f(x)\) converges for all \(x \in \kk\) with \(\|x\| = R\) if and only if \(\lim_{n \to +\infty} \|a_n\| R^n = 0\).
    % \end{proposition}
    % \begin{proof}
    %     By \cref{prop:convergence_in_ultra-metric_space}, we only need to check when the terms \(a_n x^n\) tend to zero as \(n \to +\infty\).
    %     If \(\|x\| < R\), there exists \(r \in (0,1)\) such that \(\|x\| < r^2R\).
    %     Then there exists \(N \in \bbN\) such that for all \(n \geq N\), we have \(\|a_n\|^{1/n} < 1/(rR)\) and thus
    %     \[
    %         \|a_n x^n\| = \|a_n\| \|x\|^n < \|a_n\| (r^2R)^n < (r^2R)^n \cdot \frac{1}{(rR)^n} = r^n \to 0.
    %     \]
    %     Thus the series \(f(x)\) converges for all \(x \in \kk\) with \(\|x\| < R\).

    %     Suppose that \(\|x\| > R\).
    %     There exists \(s > 1\) such that \(\|x\| > R/s\).
    %     By the definition of \(R\), there exist infinitely many \(n \in \bbN\) such that \(\|a_n\|^{1/n} > s/R\) and thus
    %     \[
    %         \|a_n x^n\| = \|a_n\| \|x\|^n > \|a_n\| \frac{R^n}{s^n} > \left(\frac{s}{R}\right)^n \cdot \frac{R^n}{s^n} = 1.
    %     \]
    %     Thus the series \(f(x)\) diverges for all \(x \in \kk\) with \(\|x\| > R\).

    %     Finally, the case \(\|x\| = R\) is direct from \cref{prop:convergence_in_ultra-metric_space}.
    % \end{proof}

    % \Yang{What about the multi-index?}


    \begin{definition}\label{def:Tate_algebra}
        Let \(\kk\) be a complete non-archimedean field.
        Let \(T = (T_1, \ldots, T_n)\) be a tuple of \(n\) indeterminates and \(r = (r_1, \ldots, r_n)\) be a tuple of \(n\) positive real numbers.
        The \emph{Tate algebra} (or \emph{ring of restricted power series}) is defined as 
        \[
            \kk\langle \underline{r^{-1}T} \rangle := \kk \{ \underline{r^{-1}T} \} := \left\{ \sum_{\alpha \in \bbN^n} a_\alpha T^\alpha \mid a_\alpha \in \kk, \lim_{|\alpha| \to +\infty} \|a_\alpha\| r^\alpha = 0 \right\}.
        \]
    \end{definition}

    \begin{proposition}\label{prop:Tate_algebra_is_a_banach_algebra_over_k}
        Let \(\kk\) be a complete non-archimedean field.
        Then the Tate algebra \(\kk\{ \underline{T/r} \} \) is a non-archimedean multiplicative banach \(\kk\)-algebra with respect to the \emph{gauss norm}
        \[
            \left\| \sum_{\alpha \in \bbN^n} a_\alpha T^\alpha \right\| := \sup_{\alpha \in \bbN^n} \|a_\alpha\|r^\alpha = \max_{\alpha \in \bbN^n} \|a_\alpha\|r^\alpha.
        \]
    \end{proposition}    
    \Yang{For the definition of banach ring, see}
    \begin{proof}
        The proof splits into several parts.
        Every parts is straightforward and standard.

        \begin{step}\label{step_in_prop:Tate_algebra_is_a_banach_algebra_over_k:k-algebra}
            We first show that \(\kk\{ \underline{T/r} \} \) is a \(\kk\)-algebra.
        \end{step}

        Easily to see that it is closed under addition and scalar multiplication.
        Suppose that \(f = \sum_{\alpha \in \bbN^n} a_\alpha T^\alpha\) and \(g = \sum_{\alpha \in \bbN^n} b_\alpha T^\alpha\) are two elements in \(\kk\{ \underline{T/r} \} \).
        Given \(\varepsilon > 0\), there exists \(N \in \bbN\) such that for all \(|\alpha| > N\), we have \(\|a_\alpha\| r^\alpha < \varepsilon/\|g\|\) and \(\|b_\alpha\| r^\alpha < \varepsilon/\|f\|\).
        For any \(|\gamma| > 2N\), we have
        \[
            \left\| \sum_{\alpha + \beta = \gamma} a_\alpha b_\beta \right\| r^\gamma \leq \max_{\alpha + \beta = \gamma} \|a_\alpha\| r^\alpha \cdot \|b_\beta\| r^\beta < \max\left\{ \frac{\varepsilon}{\|g\|} \|b_{\beta}\|r^\beta, \frac{\varepsilon}{\|f\|} \|a_{\alpha}\|r^\alpha \right\} \leq \varepsilon.
        \]
        Hence \(f \cdot g \in \kk\{ \underline{T/r} \} \) and it shows that \(\kk\{ \underline{T/r} \} \) is a \(\kk\)-algebra.

        \begin{step}\label{step_in_prop:Tate_algebra_is_a_banach_algebra_over_k:normed_k-algebra}
            Show that the gauss norm is a non-archimedean norm on \(\kk\{ \underline{T/r} \} \).
        \end{step}

        The linearity and positive-definiteness of the gauss norm are direct from the definition.
        % Suppose that \(f = \sum_{\alpha \in \bbN^n} a_\alpha T^\alpha\) and \(g = \sum_{\alpha \in \bbN^n} b_\alpha T^\alpha\) are two elements in \(\kk\{ \underline{T/r} \} \).
        We have
        \[
            \|f + g\| = \sup_{\alpha \in \bbN^n} \|a_\alpha + b_\alpha\| r^\alpha \leq \sup_{\alpha \in \bbN^n} \max\{\|a_\alpha\| + \|b_\alpha\|\} r^\alpha \leq \max\{\|f\|, \|g\|\}
        \]
        and 
        \begin{align*}
            \| f \cdot g \| &= \left\| \sum_{\gamma \in \bbN^n} \left( \sum_{\alpha + \beta = \gamma} a_\alpha b_\beta \right) T^\gamma \right\| = \sup_{\gamma \in \bbN^n} \left\| \sum_{\alpha + \beta = \gamma} a_\alpha b_\beta \right\| r^\gamma \\
            &\leq \sup_{\gamma \in \bbN^n} \max_{\alpha + \beta = \gamma} \|a_\alpha\| \|b_\beta\| r^\alpha r^\beta = \|a_{\alpha_0}\| r^{\alpha_0} \cdot \|b_{\beta_0}\| r^{\beta_0} \leq \|f\| \cdot \|g\|.
        \end{align*}
        These show that Tate algebra with the gauss norm is a non-archimedean normed \(\kk\)-algebra.

        \begin{step}\label{step_in_prop:Tate_algebra_is_a_banach_algebra_over_k:multiplicativity}
            Show that the gauss norm is multiplicative.
        \end{step}

        Suppose that \(\|f\| = \|a_{\alpha_1}\| r^{\alpha_1}\) and \(\|a_{\alpha}\|r^\alpha < \|f\|\) for all \(\alpha <_{\text{total}} \alpha_1\).
        Similar to \(\|b_{\beta_1}\| r^{\beta_1}\).
        Then we have
        \[
            \|f\| \cdot \|g\| = \|a_{\alpha_1}\| r^{\alpha_1} \cdot \|b_{\beta_1}\| r^{\beta_1} = \max_{\alpha + \beta = \alpha_1 + \beta_1} \|a_\alpha\| \|b_\beta\| r^\alpha r^\beta = \left\| \sum_{\alpha + \beta = \alpha_1 + \beta_1} a_\alpha b_\beta \right\| r^{\alpha_1 + \beta_1} \leq \| f \cdot g \|,
        \]
        where the third equality holds since \((\alpha_1, \beta_1)\) is the unique pair such that \(\|a_{\alpha_1}\| r^{\alpha_1} \cdot \|b_{\beta_1}\| r^{\beta_1}\) is maximized and by \cref{prop:all_triangles_in_ultra-metric_space_are_isosceles}.
        Thus the gauss norm is multiplicative.

        \begin{step}\label{step_in_prop:Tate_algebra_is_a_banach_algebra_over_k:completeness}
            Finally show that \(\kk\{ \underline{T/r} \} \) is complete with respect to the gauss norm.
        \end{step}

        Let \(\{f_m = \sum a_{\alpha,m}T^\alpha\}\) be a cauchy sequence in \(\kk\{ \underline{T/r} \} \).
        We have
        \[ \|a_{\alpha,m} - a_{\alpha,l}\| r^\alpha \leq \|f_m - f_l\|. \]
        Thus for each \(\alpha \in \bbN^n\), the sequence \(\{a_{\alpha,m}\}\) is a cauchy sequence in \(\kk\).
        Since \(\kk\) is complete, set \(a_\alpha := \lim_{m \to +\infty} a_{\alpha,m}\) and \(f = \sum_{\alpha \in \bbN^n} a_\alpha T^\alpha\).
        Given \(\varepsilon > 0\), there exists \(M \in \bbN\) such that for all \(m,l > M\), we have \(\|f_m - f_l\| < \varepsilon\).
        Fixing \(m > M\), there exists \(N \in \bbN\) such that for all \(|\alpha| > N\), we have \(\|a_{\alpha,m}\| r^\alpha < \varepsilon\).
        Hence for all \(|\alpha| > N\) and \(l > M\), we have
        \[ \|a_{\alpha,l}\| r^\alpha \leq \|a_{\alpha,l} - a_{\alpha,m}\| r^\alpha + \|a_{\alpha,m}\| r^\alpha < 2\varepsilon. \]
        Taking \(l \to +\infty\), we have \(\|a_\alpha\| r^\alpha \leq 2\varepsilon\) for all \(|\alpha| > N\).
        It follows that \(f \in \kk\{ \underline{T/r} \} \).

        For every \(\varepsilon > 0\), there exists \(N \in \bbN\) such that for all \(m,l > N\), we have \(\|f_m - f_l\| < \varepsilon\).
        Thus for all \(\alpha \in \bbN^n\) and \(m,l > N\), we have
        \[ \|a_{\alpha,m} - a_{\alpha,l}\| r^\alpha \leq \|f_m - f_l\| < \varepsilon. \]
        Taking \(l \to +\infty\), we have \(\|a_{\alpha,m} - a_\alpha\| r^\alpha \leq \varepsilon\) for all \(m > N\).
        It follows that
        \[ \|f - f_m\| = \sup_{\alpha \in \bbN^n} \|a_\alpha - a_{\alpha,m}\| r^\alpha \leq \varepsilon \]
        for all \(m > N\).
    \end{proof}

    \begin{proposition}\label{prop:restricted_power_series_invertible_iff_the_constant_item_controls_others}
        Let \(\kk\) be a complete non-archimedean field.
        An element \(f = \sum_{\alpha \in \bbN^n} a_\alpha T^\alpha \in \kk\{ \underline{T/r} \} \) is invertible if and only if \(\|a_0\| > \|a_\alpha\| r^\alpha\) for all \(\alpha \neq 0\).
        % \Yang{To be checked.}
    \end{proposition}
    \begin{proof}
        Multiplying by \(a_0^{-1}\), we can reduce to the case \(a_0 = 1\).
        Let \(g = \sum_{\alpha \in \bbN^n} b_\alpha T^\alpha\) be the inverse of \(f\) in \(\kk[[\underline{T}]]\).
        Then we have
        \[ f \cdot g = \sum_{\alpha \in \bbN^n} a_\alpha T^\alpha \cdot \sum_{\beta \in \bbN^n} b_\beta T^\beta = \sum_{\gamma \in \bbN^n} \left( \sum_{\alpha + \beta = \gamma} a_\alpha b_\beta \right) T^\gamma = 1. \]
        That is, for every \(\gamma \neq 0 \in \bbN^n\),
        \[ b_{\gamma} = - \sum_{\substack{\alpha + \beta = \gamma \\ \alpha \neq 0}} a_\alpha b_\beta. \]
        Let \(A = \|f-1\| < 1\).
        We show that for every \(m \in \bbN\), there exists \(C_m > 0\) such that for all \(\alpha \in \bbN^n\) with \(|\alpha| \geq C_m\), we have \(\|b_\alpha\| r^\alpha \leq A^m\).
        For \(m = 0\), note that \(b_0 = 1\).
        By induction on \(\gamma\) with respect to the total order \(\leq_{\text{total}}\), we have
        \[ \|b_\gamma\| r^\gamma \leq \max_{\substack{\alpha + \beta = \gamma \\ \alpha \neq 0}} \|a_\alpha\| r^\alpha \cdot \|b_\beta\| r^\beta \leq A \max_{\beta <_{\text{total}} \gamma} \|b_\beta\| r^\beta \leq 1. \]
        Suppose that the claim holds for \(m\).
        There exists \(D_{m+1} \in \bbN\) such that for all \(\alpha \in \bbN^n\) with \(|\alpha| \geq D_{m+1}\), we have \(\|a_\alpha\| r^\alpha \leq A^{m+1}\).
        Set \(C_{m+1} = C_m + D_{m+1} + 1\).
        For any \(\gamma \in \bbN^n\) with \(|\gamma| \geq C_{m+1}\), we have
        \[ \|b_\gamma\| r^\gamma \leq \max_{\substack{\alpha + \beta = \gamma \\ \alpha \neq 0}} \|a_\alpha\| r^\alpha \cdot \|b_\beta\| r^\beta \leq \max \left\{ A^{m+1}, A \cdot A^m \right\} = A^{m+1} \]
        since either \(|\alpha| \geq D_{m+1}\) or \(|\beta| \geq C_m\).
        Thus by induction, we have \(\|b_\alpha\| r^\alpha \to 0\) as \(|\alpha| \to +\infty\).
        It follows that \(g \in \kk\{ \underline{T/r} \} \).
    \end{proof}

    % \begin{definition}\label{def:derivative_operators}
    %     Let \(\kk\) be a complete non-archimedean field.

    % \end{definition}
    Let \(\kk\) be a complete non-archimedean field.
    Recall that a derivative operator \(\partial: \kk\{ \underline{T/r} \} \to \kk\{ \underline{T/r} \} \) is defined as the \(\kk\)-linear map such that for every multi-index \(\alpha = (\alpha_1, \ldots, \alpha_n) \in \bbN^n\), we have
    \Yang{To be revised.}

    \begin{proposition}\label{prop:formal_derivative_of_restricted_power_series_is_also_a_restricted_power_series}
        Let \(\kk\) be a complete non-archimedean field, and \(\partial\) be a derivative operator on \(\kk\{ \underline{T/r} \} \).
        Then for every \(f \in \kk\{ \underline{T/r} \} \), we have \(\partial(f) \in \kk\{ \underline{T/r} \} \).
    \end{proposition}
    \begin{proof}
        \Yang{We only need to check the case \(\partial = \partial/\partial T_1\).}
        Suppose that \(f = \sum_{\alpha \in \bbN^n} a_\alpha T^\alpha \in \kk\{ \underline{T/r} \} \).
        We have
        \[ \frac{\partial f}{\partial T_1} = \sum_{\alpha \in \bbN^n} \alpha_1 a_\alpha T_1^{\alpha_1 - 1} T_2^{\alpha_2} \cdots T_n^{\alpha_n}. \]
        Noting that \(\kk\) is non-archimedean, we have \(\|\alpha_1 a_\alpha\| \leq \|a_\alpha\|\).
        Then 
        \[ \lim_{|\alpha| \to +\infty} \| \alpha_1 a_\alpha \| r_1^{\alpha_1 - 1} r_2^{\alpha_2} \cdots r_n^{\alpha_n} \leq  \frac{1}{r_1} \lim_{|\alpha| \to +\infty} \|a_\alpha\| r^\alpha = 0. \]
        The conclusion follows.
        % \Yang{To be continued.}
    \end{proof}
